\documentclass{llncs}
\usepackage[utf8]{inputenc}
\usepackage{xspace}
\usepackage{amsmath}
\usepackage{stmaryrd}
\usepackage{marvosym}


\title{\ucftw---a Universally Composable Party Game for Two or More Parties}
\date{\today}
\author{Dirk Achenbach\inst{1} \and Brandon Broadnax\inst{1} \and Bernhard Löwe\inst{1} \and Jochen Rill\inst{2} \and Mario Strefler\inst{1}}
\institute{Karlsruhe Institute of Technology (KIT) \and FZI Forschungszentrum Informatik}

\newcommand{\ucftw}{UC$^4$TW\xspace}
\newcommand{\blitz}{\lightning}
\newcommand{\stopsign}{\Stopsign}
\newcommand{\stopcard}{\\ \vspace{2em} \Huge \stopsign}


\begin{document}
\maketitle
\begin{abstract}
	We propose \emph{Universal Composability for the Win} (\ucftw), a novel dice game for two or more parties. 
	The goal of the game is to obtain the most powerful security notion among all parties. 
	As it is often not easily determined if a security notion implies another, the key to winning the game are clever arguments.
	The game proceeds in turns. 
	Each turn, the parties throw the dice to obtain a quantifier and an entity to complement their notion. 
	Random modifiers make the game more exciting.
	The party with the most powerful security notion wins---with the obvious exception that no notion trumps the Universal Composability (UC)~\cite{can01,can05,can13} notion.
	To the best of our knowledge, \ucftw is the first UC-based game to be proposed.
\end{abstract}

\section{Introduction}
\begin{theorem}
  If a matter cannot decided by the rules of the game alone, or should any rule be ill-defined, the issue is to be resolved by discussion.
\end{theorem}

\subsection{Preliminaries}
We assume the reader is proficient in the Universal Composability (UC)~\cite{can01,can05,can13} framework.

\section{The Rules of the Game}
\ucftw is founded on three dice and a set of modifier cards.
The dice all have six sides.
The \emph{Quantifier Dice} is used to roll a quantifier, while the \emph{Entity Dice} has the sides $\mathcal{A}$, $\mathcal{S}$, and $\mathcal{Z}$.
What is more, the \emph{Modifier Dice} introduces additional modifications to the security notion.
We stress that \ucftw's mechanics do not rely on the dice rolls being uniformly distributed.
See Definition~\ref{def:dices} for a definition of the dices' faces.
\begin{definition}[The Dice]
	\begin{align*}
		\Omega_\text{Quantifier} &:= \{\exists, \forall\}  \\
		\Omega_\text{Entity} &:= \{\mathcal{A}, \mathcal{S}, \mathcal{Z}\}  \\
		\Omega_\text{Modifier} &:= \{\varepsilon, \blitz\} 
	\end{align*}
	\label{def:dices}
\end{definition}

\subsection{Setup Stage}
During the setup stage, the deck of modifier cards is shuffled.

\subsection{The Course of the Game}
\ucftw proceeds in \emph{rounds}.
During each round, the parties take turns in throwing the dice.

\begin{theorem}
	The party with the lowest publication count begins. The game continues counter-clockwise.
\end{theorem}

A party throws the dice and writes down the quantifier and entity events (in that order).


Further, if $X_\text{Modifier} = \blitz$, the party draws a card from the deck of modifier cards. 

\begin{theorem}
	Modifier cards with the \stopsign symbol can be played against \emph{any party} at any time.
	All other modifier cards immediately modify the deck of the party who picked it up. 
\end{theorem}

\begin{remark}
	A variant of the game allows modifier cards with the \stopsign symbol to be played against other parties only.
\end{remark}

\begin{definition}
	The sequence of quantifiers, entities, and modifiers a party has acquired is called its \emph{security notion} (or simply \emph{notion}).
\end{definition}

\begin{theorem}
	If any entity symbol turns up more than once in one party's notion, the occurences are to be interpreted as \emph{different} entities, e.g.\ $\mathcal{S}_1, \mathcal{S}_2$, et cetera.
\end{theorem}

\begin{theorem}
	A party retires from the game if and only if its notion contains at least one occurrence of each $\mathcal{A}$, $\mathcal{S}$, and $\mathcal{Z}$.
\end{theorem}

Finally, when all parties have retired from the game, the winner can be determined. To this end, the parties' notions are interpreted in the sense that a construction is secure if $\mathcal{Z}$'s output in the ideal model is indistinguishable from its output in the real model~\cite{can01,can05,can13}.

\begin{theorem}[Winning the Game]
	The party with the strongest security notion wins the game.
\end{theorem}

A security notion $\alpha$ is stronger than another security notion $\beta$ if $\alpha$ implies $\beta$, that is, if any protocol secure relative to $\beta$ is also secure relative to $\alpha$.

It is of course not always self-evident whether a notion implies another. Such cases of ambiguity are to be resolved by a discussion among the parties. If no direct implication can be found, the parties must agree on a weaker form of implication. What is more, it can be counted as an added bonus if a notion has a Composition Theorem.

\begin{theorem}[Weakly Winning the Game]
	If no strongest security notion can be determined, the party with the best security notion wins the game.
\end{theorem}

Another concern \ucftw addresses is the distinctiveness of the UC notion~\cite{can01,can05,can13}.

\begin{theorem}[The Superiority Rule]
	A party with the notion $\forall\mathcal{A}\exists\mathcal{S}\forall\mathcal{Z}$ instantly wins the game, even before the other parties retire from the game.
\end{theorem}

\section{Future Work}
Actually playing the game is outside the scope of this work.
We leave it an open research question to evaluate the game's mechanics.

\begin{thebibliography}{9}
\bibitem{can01}
  Ran Canetti,
  \emph{Universally Composable Security: A New Paradigm for Cryptographic Protocols},
  Cryptology ePrint Archive: Report 2000/067,
  2001.

\bibitem{can05}
  Ran Canetti,
  \emph{Universally Composable Security: A New Paradigm for Cryptographic Protocols},
  Cryptology ePrint Archive: Report 2000/067,
  2005.

\bibitem{can13}
  Ran Canetti,
  \emph{Universally Composable Security: A New Paradigm for Cryptographic Protocols},
  Cryptology ePrint Archive: Report 2000/067,
  2013.
\end{thebibliography}
\end{document}
