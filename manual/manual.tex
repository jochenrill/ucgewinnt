\documentclass{llncs}
\usepackage[utf8]{inputenc}
\usepackage{xspace}
\usepackage{amsmath}
\usepackage{stmaryrd}
\usepackage{marvosym}


\title{\ucftw---a Universally Composable Dice Game for Two or More Players}
\date{\today}
\author{Dirk Achenbach\inst{1} \and Brandon Broadnax\inst{1} \and Bernhard Löwe\inst{1} \and Jochen Rill\inst{2} \and Mario Strefler\inst{1}}
\institute{Karlsruhe Institute of Technology (KIT) \and FZI Forschungszentrum Informatik}

\newcommand{\ucftw}{UC$^4$TW\xspace}
\newcommand{\blitz}{\lightning}
\newcommand{\stopsign}{\Stopsign}
\newcommand{\stopcard}{\\ \vspace{2em} \Huge \stopsign}


\begin{document}
\maketitle
\begin{abstract}
	We propose \emph{Universal Composability for the Win} (\ucftw), a novel dice game for two or more players. 
	The goal of the game is to obtain the most powerful security notion among the players. 
	As it is often not easily determined if a security notion implies another, the key to winning the game are clever arguments.
	The game proceeds in turns. 
	Each turn, the players throw the dice to obtain a quantifier and an entity to complement their notion. 
	Random modifiers spice things up.
	The player with the most powerful security notion wins---with the obvious exception that no notion trumps the Universal Composability (UC)~\cite{can01,can05,can13} notion.
	To the best of our knowledge, \ucftw is the first UC-based game to be proposed.
\end{abstract}

\section{Introduction}
\begin{theorem}
  If a matter cannot decided by the rules of the game alone, or should any rule be ill-defined, the issue is to be resolved by discussion.
\end{theorem}

\subsection{Prior Work}

\section{The Rules of the Game}
\ucftw is founded on three dice and a set of modifier cards.
The dice all have six sides.
The \emph{Quantifier Dice} is used to roll a quantifier, while the \emph{Entity Dice} has the sides $\mathcal{A}$, $\mathcal{S}$, and $\mathcal{Z}$.
What is more, the \emph{Modifier Dice} introduces additional modifications to the security notion.
We stress that \ucftw's mechanics do not rely on the dice rolls being uniformly distributed.
See Definition~\ref{def:dices} for a definition of the dice's faces.
\begin{definition}[The Dice]
	\begin{align*}
		\Omega_\text{Quantifier} &:= \{\exists, \forall\}  \\
		\Omega_\text{Entity} &:= \{\mathcal{A}, \mathcal{S}, \mathcal{Z}\}  \\
		\Omega_\text{Modifier} &:= \{\varepsilon, \blitz\} 
	\end{align*}
	\label{def:dices}
\end{definition}

\subsection{Setup Stage}
During the setup stage, the deck of modifier cards is shuffled.

\subsection{The Course of the Game}
\ucftw proceeds in \emph{rounds}.
During each round, the players take turns in throwing the dice.
A player throws the dice and writes the quantifier and entity event down.
Further, if $X_\text{Modifier} = \blitz$, the player draws a card from the deck of modifier cards. 

\begin{theorem}
	The player with the lowest publication count begins. The game continues counter-clockwise.
\end{theorem}

\stopsign

\section{Future Work}
Actually playing the game was outside the scope of this work.
We leave evaluating the game's mechanics open for future work.

\begin{thebibliography}{9}
\bibitem{can01}
  Ran Canetti,
  \emph{Universally Composable Security: A New Paradigm for Cryptographic Protocols},
  Cryptology ePrint Archive: Report 2000/067,
  2001.

\bibitem{can05}
  Ran Canetti,
  \emph{Universally Composable Security: A New Paradigm for Cryptographic Protocols},
  Cryptology ePrint Archive: Report 2000/067,
  2005.

\bibitem{can13}
  Ran Canetti,
  \emph{Universally Composable Security: A New Paradigm for Cryptographic Protocols},
  Cryptology ePrint Archive: Report 2000/067,
  2013.
\end{thebibliography}
\end{document}
