\documentclass{llncs}
\usepackage[utf8]{inputenc}
\usepackage{xspace}

\title{\ucftw---a Universally Composable Dice Game for Two or More Players}
\date{\today}
\author{Dirk Achenbach\inst{1} \and Brandon Broadnax\inst{1} \and Bernhard Löwe\inst{1} \and Jochen Rill\inst{2} \and Mario Strefler\inst{1}}
\institute{Karlsruhe Institute of Technology (KIT) \and FZI Forschungszentrum Informatik}

\newcommand{\ucftw}{UC$^4$TW\xspace}

\begin{document}
\maketitle
\begin{abstract}
	We propose \emph{Universal Composability for the Win} (\ucftw), a novel dice game for two or more players. 
	The goal of the game is to obtain the most powerful security notion at the table. 
	As it is often not easily determined if a security notion implies another, the key to winning the game are clever arguments.
	The game proceeds in turns. 
	Each turn, the players throw the dice to obtain a quantifier and a party to complement their notion. 
	Random modifiers spice things up.
	The player with the most powerful security notion wins---with the obvious exception that no notion trumps the Universal Composability (UC)~\cite{can01,can05,can13} notion.
	To the best of our knowledge, \ucftw is the first UC-based game to be proposed.
\end{abstract}

\section{Introduction}
\begin{theorem}
  If a matter cannot decided by the rules of the game alone, or should any rule be ill-defined, the issue is to be resolved by discussion.
\end{theorem}

\subsection{Prior Work}

\section{Rules of the Game}
\subsection{The Dice}
\subsection{Modifier Cards}

\section{Future Work}
We leave actually playing the game open for future work.

\begin{thebibliography}{9}
\bibitem{can01}
  Ran Canetti,
  \emph{Universally Composable Security: A New Paradigm for Cryptographic Protocols},
  Cryptology ePrint Archive: Report 2000/067,
  2001.

\bibitem{can05}
  Ran Canetti,
  \emph{Universally Composable Security: A New Paradigm for Cryptographic Protocols},
  Cryptology ePrint Archive: Report 2000/067,
  2005.

\bibitem{can13}
  Ran Canetti,
  \emph{Universally Composable Security: A New Paradigm for Cryptographic Protocols},
  Cryptology ePrint Archive: Report 2000/067,
  2013.
\end{thebibliography}
\end{document}
